\documentclass[10pt,letterpaper]{article}
\usepackage[utf8]{inputenc}
\usepackage{amsmath}
\usepackage{amsfonts}
\usepackage{amssymb}
\author{Christopher C. Lamb}
\title{Reference Generator}
\begin{document}

\section{References and Summaries from Cloud Robotics}

\paragraph{Cloud Networked Robotics.} In 2012, Kamei et. al. proposed establishing a field of {\sl Cloud Networked Robotics}.  While not directly concerned with using cloud computing in the context of autonomous systems, Kamei and colleagues present a case supporting swarms of robots with distinct functions that collaborate to effectively support individuals.  They do however describe the need to provide distributed computation and data storage services to swarms of robots enabling information sharing and coordinated planning.  Key needs they identify for distributed computing systems include providing abstractions for individual robots, so a swarm can be automatically coordinate actions toward a goal rather than requiring programming for single entities, and the ability to share different types of information freely among cooperating robots~\cite{KaNiHaSa:12}.

\paragraph{Cloud Robotics: Current trends and possible use as a service.} Lorencik in 2013 outlined the state of the art in cloud systems supporting robotics deployments, outlining advantages to using cloud computing with robots in cost and performance domains, at the expense of network connection and latency dependence~\cite{LoSi:13}.

\paragraph{Robot as a Service in Cloud Computing.}

\cite{YiZhGa:10}

\paragraph{Cloud Robotics: Towards Context Aware Robotic Networks}

\cite{QuMeDi:11}

\paragraph{Cloud Computing Environment for Supporting Network Robotics Applications.}

\cite{AgOlFePa:11}

\paragraph{DAvinCi: A Cloud Computing Framework for Service Robots.}

\cite{ArEnLiWu:10}

\bibliographystyle{unsrt}
\bibliography{cloud-robotics}
\end{document}