\documentclass[10pt,letterpaper]{article}
\usepackage[utf8]{inputenc}
\usepackage{amsmath}
\usepackage{amsfonts}
\usepackage{amssymb}
\author{Christopher C. Lamb}
\title{Reference Generator}
\begin{document}

\section{References and Summaries from Cloud Robotics}

\paragraph{Cloud Networked Robotics.} In 2012, Kamei et. al. proposed establishing a field of {\sl Cloud Networked Robotics}.  While not directly concerned with using cloud computing in the context of autonomous systems, Kamei and colleagues present a case supporting swarms of robots with distinct functions that collaborate to effectively support individuals.  They do however describe the need to provide distributed computation and data storage services to swarms of robots enabling information sharing and coordinated planning.  Key needs they identify for distributed computing systems include providing abstractions for individual robots, so a swarm can be automatically coordinate actions toward a goal rather than requiring programming for single entities, and the ability to share different types of information freely among cooperating robots \cite{KaNiHaSa:12}.

\paragraph{Cloud Robotics: Current trends and possible use as a service.} Lorencik in 2013 outlined the state of the art in cloud systems supporting robotics deployments, outlining advantages to using cloud computing with robots in cost and performance domains, at the expense of network connection and latency dependence \cite{LoSi:13}.

\paragraph{Cloud Robotics: Architecture, Challenges, and Applications}
In 2012, Hu outlined both two primary patterns of communication within cloud robotics as well as rigorous guidelines. based on power demands, on when they should be used and why.  He also presented communication protocols useful in both scenarios, and presented communication performance expectations.  Overall, this thorough evaluation characterizes computation based on power utilization well, but does not address other areas of concern, like urgency and result latency.  Likewise, the structural communication patterns presented are well analyzed with respect to communication performance and information distribution, but those patterns are not yet addressed from other perspectives, like urgency, or data size, nor are specific uses motivated \cite{HuWeYo:12}.

\paragraph{Robot as a Service in Cloud Computing.}
Chen proposed viewing robots as a service withing deployed constellations in 2011.  In this model, which he did implement, he developed clear interfaces to robots using web-centric technologies in which robots could be controlled as a service rather than as a specific robotic entity. The implementation follows accepted service-oriented-architecture principles well, treating each robot as a service unit with specific services published within a larger service directory.  These services can then be aggregated into specific applications within a given robotic constellation.  Chen developed and implemented the original robot as a service model, but as of yet has not provided analytical guidance on when this kind of model is appropriate, exactly why it should be used, and how it could be implemented at scale in the presence of  \cite{YiZhGa:10}.

\paragraph{Cloud Robotics: Towards Context Aware Robotic Networks}
Quintas, in 2011, outlined and implemented a hybrid system of smart room sensors and robots integrated via message-based service oriented systems.  This use of service oriented architecture (SOA) is closely related to cloud computing systems, and though it does not display the provisioning or scale of cloud computing in the outlined system, implementation of the services on top of a cloud provider's infrastructure would certainly address this gap.  Overall, the proposed sytem is both well designed and reflects common SOA services including service discovery, and demonstrates the applicability of this general class of architectures to cloud robotics.  It does not however provide any guidance with regard to when this kind of system should be used, or how it should enhance the cost or performance of integrated robotic systems outside of implementing the Trinity Scenario \cite{QuMeDi:11}.

\paragraph{A Cloud Computing Environment for Supporting Network Robotics Applications.}

\cite{AgOlFePa:11}

\paragraph{DAvinCi: A Cloud Computing Framework for Service Robots.}

\cite{ArEnLiWu:10}

\bibliographystyle{unsrt}
\bibliography{cloud-robotics}
\end{document}